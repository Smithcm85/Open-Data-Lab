A key component of openness is making resources usable. This idea falls in line with the idea that openness and accessibility are part of the same mission. As a result the educational component of the Open Data Lab is vital to the suces of the project. There are two main thrusts in this endeavor. The first, which was piloted in 2018, is the production of educational materials and methods. The second part is the development of communication paradigms.
This year two workshops were produced and delivered as part of the closed beta test. The first focused on the scale data protocol spark and the second the version control tool GitHub.

\section{Spark Workshop}
This workshop was designed as an introduction to spark. The goals were:
\begin{itemize}
\item Teach how to get started
\item Build comfort
\item Teach how to get answers to further questions
\end{itemize}
The topics covered included linking to a spark context, reading in data via dataframes, manipulating the data, and making a fundamental calculation.
To power the workshop the attendees were given credentials on a Amazon SageMaker notebook. One of the features of this approach is that the whole workshop has access to the same environment. Everyone sees the same implementation of the software and hardware. There is no cumbersome overhead in getting set up. The requirements are a web browser and access to the internet.
Furthermore the single notebook environment leads to a very useful pedagocial capability. When a student encounters and error in their code the instructor can load their notebook on the main display in the room. In real time and in full view of everyone the instructor can debug and teach the whole class. This is a vast improvement over the current popular method of hovering over a single learners station. It enables every person in the room to see what is going on and maintain their level of engagement. This experience was very positive for the learners and the feedback to this approach was superlative.
Previous versions of spark training was done with Databricks resources and there were several drawbacks precipitating the switch to Amazon.
\begin{itemize}
\item The environment is not shared between the workshop participants and the instructor
\item Every learner independently established their own cluster and there is substantial lag
\item Materials must be imported in Databricks format (.dbc) instead of more universal jupyter notebook format (.ipynb).
\end{itemize}

Resources:
\begin{itemize}
\item Databricks based workshop can be found at: https://github.com/alonzi/spark
\item Amazon Sakemaker based workshop can be found at: https://github.com/alonzi/spark-intro
\item Next generation materials will be incorporated into the Open Data Lab repository at: https://github.com/UVA-DSI/Open-Data-Lab/tree/master/education
\end{itemize}

The cost to operate and instructional environment is \$0.0464 per hour. For this workshop we ran the environment for one week at a cost of approximately \$10.

\section{GitHub Workshop}
\label{sec:git}
The Open Data Lab was invited to present GitHub to the Archaeology Department of the Thomas Jefferson Foundation (aka Monticello). We developed a workshop to explain the fundamentals of version control and present a work-flow for beginning users. The different user archetypes were also discussed.
One of the major burdens to version control use is that it comes from the computer superuser community. Most of the software is developed using a terminal based interlace (CLI). However in today's research world many one computer superusers interact with code and other materials that benefit from a version control workflow.
The major benefit from GitHub is the browser based interface. This implementation shifts substantial pieces of cognative load off the user. This shift enables the user to focus on developing their work rather than on the bookkeeping of version controling their work. At the same time it makes it easy for the developers to take advantage of the version control benefits.
There is substantial room to further develop materials for different user archeytypes. This workshop focused on a research group. We will strive to identify other archetypes and develop materials to suit those needs.
This workshop was taught straight from the GitHub repository itself. That was a natural fit given the subject matter. But it also demonstrated several very useful pieces of GitHub as a teaching medium, which will be discussed in~\ref{githubforteaching}.

Resources: https://github.com/UVA-DSI/Open-Data-Lab/tree/master/education/GitHub

\section{Using GitHub as a Teaching Medium}
\label{githubforteaching}
Both of the workshops taught under the Open Data Lab project used GitHub as the repository for materials. This has several benefits.
\begin{itemize}
\item GitHub provides a URL and free hosting for resources
\item Subsequent changes to the materials are stored under version control thus allowing the actual materials presented to be recovered
\item Any learner who wants to suggest improvements to materials can implement a pull request
\end{itemize}

The decision to put the materials in GitHub was one of necessity since the Open Data Lab GitHub page serves as the repository for all Open Data Lab resources. GitHub by default provides a URL for every item stored in the repository and presents the README file of a repo. This presentation is automatically rendered from markdown. Wikipedia has demonstrated the success of using markdown for content presentation but to enumerate some key features here. The document is organized, hyperlinkable, figures are easily embeded, and it seamlessly renders text alongside code and mathematical formulae.

\section{Plans for 2019}
The Open Data Lab is scheduled to teach several workshops in 2019:
\begin{itemize}
\item A five part series on Spark for the DSI MSDS cohort
\item A three part series for the UVA Library on Python and Machine Learning
\item Going into the Fall 2019 semester the teaching efforts will be focused on so-called booster shots for the MSDS program.
\end{itemize}

The past year has yielded several findings in the education space. The best one to act on going forward is the effectiveness of self-contained topical seminars. To that end the Open Data Lab will focus on producing a bootcamp startup series for the 2019-2020 MSDS cohort and several booster shot seminars throughout the  term. This will cover topics including: R, Python, GitHub, BASH, programming, analysis, cloud computing, and more.