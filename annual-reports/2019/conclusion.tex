The 2018-2019 year was the first year for the Open Data Lab. In this time we learned many things and many users attended workshops, hosted and anlyzed data, some event produced a publication. We served 116 users, working on 25 projects, and 6 individuals contributed directly to the ODL project. Going forward we look to increase all of those numbers including users from beyond the Data Science Institute.

Here are a few highlights from 2018-2019:
\begin{itemize}
\item Development of a GitHub workflow for beginners useful beyond the hard sciences. This workflow was adopted by an international collaboration called the Open Greek and Latin Project~\cite{ref:oglp} as well as the Archaeology Department of Monticello~\cite{ref:tjf}. Details are found on our github page and in Section~\ref{sec:git}.
\item We realized the power of Project Jupyter as a system to deliver resources to the user without excessive cognitive load. This platform is transformative and will lead to great things in the future. Details are found in Section~\ref{sec:projectjupyter}.
\item We studied three user archetypes for the Open Data Lab: the collaborator, the student, and the sharer. Details are round in Section~\ref{sec:archetypes}.
\end{itemize}

We have accomplished a lot but now we need help. If you have an idea or want to join the team please reach out by emailing datascientist@virginia.edu (Subject Line: "I want to help the ODL").

% \footnote{https://www.dh.uni-leipzig.de/wo/projects/open-greek-and-latin-project/}
% \footnote{https://www.monticello.org/site/research-and-collections/monticello-archaeology}