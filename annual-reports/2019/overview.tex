\section{What is the Open Data Lab?}
\subsection*{OPEN}
We encourage all users to be as open as possible with every aspect of their work. That may be in opening up their data sets, publication, source code, or something else.
Data Science Institute working definition of Open:
\begin{quote}
Openness means team members responsibly sharing their data and professional endeavors (when possible and ethical). We believe in the importance of practicing openness because advancement requires assembling a heap of known pieces into a coherent picture containing new knowledge. In the world today some of the necessary pieces are unknown due to traditional non-open information practices. Wide spread open practices are the first steps to changing the world.
\end{quote}

This definition was developed by the DSI open working group. Their phase one summary is listed in Appendix~\ref{chap:owg}.

\subsection*{DATA}
We take an expansive definition of data. Everything from traditional data, to code, to workflows, to published material, and so on is considered data to us. We provide a place for all things digital data.
\subsection*{LAB}
We provide a place where the power of computing can be brought to bear against data resources. Given the scale of data today this means colocating the data and computational resources.
\subsection*{Open Data Lab}
The Open Data Lab is a resource to provide state of the art computing and data infrastructure to researchers, students, and sharers. It is guided by the principles of science and openness.
 
\section{User Archetypes}
\label{sec:archetypes}
There are many potential use cases for the Open Data Lab. In this section we describe the three cases that have been tested so far. They are: the Collaborator, someone who is working on a research project; the Student, someone who is using the Open Data Lab to learn; and the Sharer, a person with data who wants to open it up to a broader audience.

\subsection*{The Collaborator}
This archetypal person uses the Open Data Lab to conduct research. They access data and computational resources that are colocated. This colocation facilitates lower latency and increased performance. A wide range of services can be provided globally by AWS and locally through UVA HPC resources. Sample workflow:
\begin{enumerate}
\item Request a user account on the Open Data Lab
\item Once per collaboration:
\begin{enumerate}
\item Load data 
\item Provision computational resources
\end{enumerate}
\item Conduct research operations
\item Register resulting products in Dataverse
\end{enumerate}

\subsection*{The Student}
This archetype uses the Open Data Lab to facilitate learning. An example would be someone who participates in a workshop where and ODL notebook instance powered by AWS SageMaker provides the working environment. Sample workflow:
\begin{enumerate}
\item Request a user account on the Open Data Lab
\item Logon to AWS console to launch Jupyter
\item Use Jupyter during the workshop
\end{enumerate}

\subsection*{The Sharer}
This archetype is a user who owns data and wants to make it available. There are many mechanisms for sharing the data ranging from RESTful API of S3, to a SPARQL endpoint. Sample workflow:
\begin{enumerate}
\item Request a user account on the Open Data Lab
\item Load data into an S3 bucket
\item Configure one of the following
\begin{enumerate}
\item SPARQL endpoint
\item API Gateway to access S3
\item S3 permissions for a SageMaker notebook
\item ...
\end{enumerate}
\end{enumerate}

\section{User Summary}
\begin{center}
\begin{tabular}{lccr}
\hline
\hline
group & projectID & \# members & type \\
\hline
\hline
Bourne-Mura & bamc & 4 & MSDS Capstone \\
CBW & cbwc & 4 & MSDS Capstone\\
Wiki & wiki & 9 & MSDS Capstone\\
Mental Health & miip & 6 & SYS Capstone\\
\hline
Women Terror Recruitment & watr & 2 & Presidential Fellow\\
\hline
Healthy Markets & hmtt & 5 & DSI Research\\
Independent Study & pmis & 1 & DSI Research\\
\hline
Linked Open Data   & nept & 2 & External Data\\
\hline
Spark & sprk & 17 & Education \\
GitHub & gith & 9 & Education \\
\hline
ORCI & orci & 2 & ODL Development\\
\hline
ML under & mlunder & 7 & Club\\
ML grad & mlgrad & 3 & Club\\
\hline
Rivanna & -- & 11 & Local\\
Ivy & -- & 6 & Local\\
\hline
\hline
ODL-education  & -- & 26 & Education Users\\
ODL-users & -- & 68 & Unique Users\\
\hline
\hline
\end{tabular}
\end{center}

\section{A phased approach}
\label{phases}
The first three phases of the Open Data Lab have been outlined. Phase 0 focused on pre investigation and decided on what technology to test in Phase 1, the closed beta. Phase 2 is an open Beta and will serve the community of Charlottesville and other associated research and educational efforts.
\begin{center}
\begin{tabular}{lccr}
\hline
\hline
Phase & type & start & end \\
\hline
0 & alpha & FEB 2018 & JUN 2018\\
1 & closed beta & JUL 2018 & MAY 2019 \\
2 & open beta & TBD & -- \\
\hline
\hline
\end{tabular}
\end{center}
To realize phase 2 we need a person to take ownership of each element of the Open Data Lab that moves into phase 2. We also require a personnel roster capable of providing service at the level necessary for the users. That requires hiring and as a results the timeline is TBD.

Put another way, the current bus factor\footnote{\url{https://en.wikipedia.org/wiki/Bus_factor}} for the open data lab is 1.

\subsection*{Long Term Goals}
\begin{enumerate}
\item A sustainable co-located platform for research
\item Users receiving grants they could not have received without the ODL
\item A training ground for the next generation of scientists
\end{enumerate}

\subsection*{Short Term Goals}
\begin{enumerate}
\item Explore technology to solve open Open Data Lab needs
\item Serve Data Science Institute users
\item Further understand user archetypes and innovate to serve them
\end{enumerate}

\section{What's next for the Open Data Lab?}
The year 2020 will be a big year for the Open Data Lab. The gathering of information and skill from the closed beta has gone well. We have identified areas where we can make an impact and are ready to support the research community. While continuing the closed beta through the spring semester 2019 the Open Data Lab will be seeking to add staff to prepare for the open beta launch. Those additional staff will take ownership of a particular facet of the project and lead the team in that endeavor.